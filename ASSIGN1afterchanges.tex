\documentclass[journal,12pt,twocolumn]{IEEEtran}

\usepackage{setspace}
\usepackage{gensymb}

\singlespacing


\usepackage[cmex10]{amsmath}

\usepackage{amsthm}
\usepackage{amsmath}
\usepackage{mathrsfs}
\usepackage{txfonts}
\usepackage{stfloats}
\usepackage{bm}
\usepackage{cite}
\usepackage{cases}
\usepackage{subfig}


\usepackage{longtable}
\usepackage{multirow}

\usepackage{enumitem}
\usepackage{mathtools}
\usepackage{steinmetz}
\usepackage{tikz}
\usepackage{circuitikz}
\usepackage{verbatim}
\usepackage{tfrupee}
\usepackage[breaklinks=true]{hyperref}
\raggedbottom

\usepackage{tkz-euclide}

\usetikzlibrary{calc,math}
\usepackage{listings}
    \usepackage{color}                                            %%
    \usepackage{array}                                            %%
    \usepackage{longtable}                                        %%
    \usepackage{calc}                                             %%
    \usepackage{multirow}                                         %%
    \usepackage{hhline}                                           %%
    \usepackage{ifthen}                                           %%
    \usepackage{lscape}     
\usepackage{multicol}
\usepackage{chngcntr}
\graphicspath{ {./images/} }

\DeclareMathOperator*{\Res}{Res}

\renewcommand\thesection{\arabic{section}}
\renewcommand\thesubsection{\thesection.\arabic{subsection}}
\renewcommand\thesubsubsection{\thesubsection.\arabic{subsubsection}}

\renewcommand\thesectiondis{\arabic{section}}
\renewcommand\thesubsectiondis{\thesectiondis.\arabic{subsection}}
\renewcommand\thesubsubsectiondis{\thesubsectiondis.\arabic{subsubsection}}


\hyphenation{op-tical net-works semi-conduc-tor}
\def\inputGnumericTable{}                                 %%

\lstset{
%language=C,
frame=single, 
breaklines=true,
columns=fullflexible
}
\begin{document}


\newtheorem{theorem}{Theorem}[section]
\newtheorem{problem}{Problem}
\newtheorem{proposition}{Proposition}[section]
\newtheorem{lemma}{Lemma}[section]
\newtheorem{corollary}[theorem]{Corollary}
\newtheorem{example}{Example}[section]
\newtheorem{definition}[problem]{Definition}

\newcommand{\BEQA}{\begin{eqnarray}}
\newcommand{\EEQA}{\end{eqnarray}}
\newcommand{\define}{\stackrel{\triangle}{=}}
\bibliographystyle{IEEEtran}
\providecommand{\mbf}{\mathbf}
\providecommand{\pr}[1]{\ensuremath{\Pr\left(#1\right)}}
\providecommand{\qfunc}[1]{\ensuremath{Q\left(#1\right)}}
\providecommand{\sbrak}[1]{\ensuremath{{}\left[#1\right]}}
\providecommand{\lsbrak}[1]{\ensuremath{{}\left[#1\right.}}
\providecommand{\rsbrak}[1]{\ensuremath{{}\left.#1\right]}}
\providecommand{\brak}[1]{\ensuremath{\left(#1\right)}}
\providecommand{\lbrak}[1]{\ensuremath{\left(#1\right.}}
\providecommand{\rbrak}[1]{\ensuremath{\left.#1\right)}}
\providecommand{\cbrak}[1]{\ensuremath{\left\{#1\right\}}}
\providecommand{\lcbrak}[1]{\ensuremath{\left\{#1\right.}}
\providecommand{\rcbrak}[1]{\ensuremath{\left.#1\right\}}}
\theoremstyle{remark}
\newtheorem{rem}{Remark}
\newcommand{\sgn}{\mathop{\mathrm{sgn}}}
\providecommand{\abs}[1]{\left\vert#1\right\vert}
\providecommand{\res}[1]{\Res\displaylimits_{#1}} 
\providecommand{\norm}[1]{\left\lVert#1\right\rVert}
%\providecommand{\norm}[1]{\lVert#1\rVert}
\providecommand{\mtx}[1]{\mathbf{#1}}
\providecommand{\mean}[1]{E\left[ #1 \right]}
\providecommand{\fourier}{\overset{\mathcal{F}}{ \rightleftharpoons}}
%\providecommand{\hilbert}{\overset{\mathcal{H}}{ \rightleftharpoons}}
\providecommand{\system}{\overset{\mathcal{H}}{ \longleftrightarrow}}
	%\newcommand{\solution}[2]{\textbf{Solution:}{#1}}
\newcommand{\solution}{\noindent \textbf{Solution: }}
\newcommand{\cosec}{\,\text{cosec}\,}
\providecommand{\dec}[2]{\ensuremath{\overset{#1}{\underset{#2}{\gtrless}}}}
\newcommand{\myvec}[1]{\ensuremath{\begin{pmatrix}#1\end{pmatrix}}}
\newcommand{\mydet}[1]{\ensuremath{\begin{vmatrix}#1\end{vmatrix}}}
\numberwithin{equation}{subsection}
\makeatletter
\@addtoreset{figure}{problem}
\makeatother
\let\StandardTheFigure\thefigure
\let\vec\mathbf
\renewcommand{\thefigure}{\theproblem}
\def\putbox#1#2#3{\makebox[0in][l]{\makebox[#1][l]{}\raisebox{\baselineskip}[0in][0in]{\raisebox{#2}[0in][0in]{#3}}}}
     \def\rightbox#1{\makebox[0in][r]{#1}}
     \def\centbox#1{\makebox[0in]{#1}}
     \def\topbox#1{\raisebox{-\baselineskip}[0in][0in]{#1}}
     \def\midbox#1{\raisebox{-0.5\baselineskip}[0in][0in]{#1}}
\vspace{3cm}

\title{ASSIGNMENT 1}
\author{NSV SARATH CHANDRA(CC20MTECH14001)}
\maketitle
\newpage
\bigskip
\renewcommand{\thefigure}{\theenumi}
\renewcommand{\thetable}{\theenumi}

	
\section{Problem}
Check which of the following are solutions of the following equation:
\begin{align} \myvec{1 -2}x = 4\end{align}
% Find the coordinates of the point where the line through$ \myvec{3 \\-4 \\-5}$ and $\myvec{2 \\-3 \\1}$ crosses the plane \begin{align}\myvec{2 & 1 & 1}x=7 \end{align}

\section{Explanation}\label{Explanation}

A point C lying on the line 
\begin{align}
    \myvec{a & b}x = d
\end{align}
At any distance $\lambda$ from point x lying on the same line is given as 

\begin{align}
    c = x + \frac{\lambda}{\sqrt{a^2+b^2}}\myvec{b \\ -a}
\end{align}

We have $\lambda = \sqrt{a^2+b^2}$ 
\implies c = x + \myvec{b \\ -a}

\section{Solution}


Equation of y axis is 
\begin{align}
    \myvec{1 & 0}x = 0
\end{align}

For   $\myvec{
1 & -2 \\
}x = 4$ (at y axis meet)

\begin{align}
    \myvec{
1 & -2\\
1 & 0 
}y_1 =  \myvec{ 4\\
0
}
\end{align}

\begin{align}
    y_1 =  \myvec{
1 & -2\\
1 & 0 
}^{-1}\myvec{ 4\\
0
}
\end{align}
\begin{align}
    y_1 =  \myvec{ 0\\
-2
}
\end{align}

Another point $c_1$ on the line is found using 
\begin{align}
c_1 = y_1 + \myvec{ -2\\
-1
}
\end{align}

\begin{align}
    = \myvec{ 0\\
-2
} + \myvec{ -2\\
-1
} = \myvec{ -2\\
-3
}
\end{align}

Equation for x axis is $\myvec{1 & 0}y=0$

\begin{align}
    \myvec{
1 & -2\\
0 & 1 
}y =  \myvec{ 4\\
0
}
\end{align}

\begin{align}
    y =  \myvec{
1 & -2\\0 & 1 
}^{-1}\myvec{ 4\\
0
}
\end{align}

\begin{align}
    y = 
\myvec{ 4\\
0
}
\end{align}


\section{Checking for Additional Solutions}


\begin{enumerate}
  \item For $x = \myvec{ 0\\
2
}$, we have \\$\myvec{1 & -2}x = 1*0+-2*2=-4!=4$\\

Hence not a solution. \\

 \item For $x = \myvec{ 2\\
0
}$, we have \\$\myvec{1 & -2}x = 1*2+-2*0=2!=4$\\

Hence not a solution.  \\

\item For $x = \myvec{ 1\\
1
}$, we have \\$\myvec{1 & -2}x = 1*1+-2*1=-1!=4$\\

Hence not a solution. \\

\item For $x = \myvec{ \sqrt{2}\\
4\sqrt{2}
}$, we have \\$\myvec{1 & -2}x = 1*\sqrt{2}+-2*4\sqrt{2}=-7\sqrt{2} !=4$\\

Hence not a solution. 
\end{enumerate}\\


Therefore, solution is  only \myvec{4\\0}



\end{document}